\documentclass[]{report}

\voffset=-1.5cm
\oddsidemargin=0.0cm
\textwidth = 480pt

\usepackage{framed}
\usepackage{subfiles}
\usepackage{graphics}
\usepackage{newlfont}
\usepackage{eurosym}
\usepackage{amsmath,amsthm,amsfonts}
\usepackage{amsmath}
\usepackage{color}
\usepackage{amssymb}
\usepackage{multicol}
\usepackage[dvipsnames]{xcolor}
\usepackage{graphicx}
\begin{document}


\begin{enumerate}
%%%%%%%%%%%%%%%%%%%%%%%%%%%%%%%%%%%%%%%%%%%%%%%%%%
\item 

Suppose $A = \{1,2,3,4\}$. Consider the following relation in A

\[ \{  (1,1),(2,2),(2,3),(3,2),(4,2),(4,4)\} \]

Draw the direct graph of $A$. Based on the Digraph of $A$ discuss whether or not a relation that could be depicted by the digraph could be described as the following, justifying your answer.

\begin{multicols}{2}
\begin{itemize}
\item[(a)] Symmetric
\item[(b)] Reflexive 
\item[(c)] Transitive
\item[(d)] Antisymmetric
\end{itemize}
\end{multicols}
%---------------- %
\item Let S be a set and R be a relation on S. Explain what it means to say that R
is
\begin{multicols}{3}
\begin{itemize}
\item[(a)] reflexive,
\item[(b)] symmetric,
\item[(c)] transitive,
\item[(d)] anti-symmetric,
\item[(e)] an equivalence relation,
\item[(f)] a partial order,
\item[(g)] an order. 
\end{itemize}
\end{multicols}


\item 
Determine which of the following relations $ x R y$ are reflexive, transitive, symmetric, or antisymmetric on the following - there may be more than one characteristic. 
\begin{multicols}{2}
\begin{itemize} 
\item[(a)] $x = y$
\item[(b)] $x < y$
\item[(c)] $x^2 = y^2$
\item[(d)] $x \geq y$
\end{itemize}
\end{multicols}
%---------------------------------
\item 
% 2007 Q8
Given a flock of chickens, between any two chickens one of them is
dominant. A relation, R, is defined between chicken x and chicken y as xRy if x is
dominant over y. This gives what is known as a pecking order to the flock. Home
Farm has 5 chickens: Amy, Beth, Carol, Daisy and Eve, with the following relations:

\begin{itemize}
\item Amy is dominant over Beth and Carol
\item Beth is dominant over Eve and Carol
\item Carol is dominant over Eve and Daisy
\item Daisy is dominant over Eve, Amy and Beth
\item Eve is dominant over Amy.
\end{itemize}

%% REST OF QUESTIOM

\item 
For $a, b \in Z$ define a R b to mean that a divides b.

\begin{itemize}
\item[(a)] State whether or not R is reflexive.
\item[(b)] State whether or not R is symmetric.
\item[(c)] State whether or not R is transitive.
\end{itemize}
% Solution: 
% 
% \begin{itemize}
% \item[(a)]  Since 0 does not divide 0, therefore R is not reflexive.
% \item[(b)] 2 divides 4 so 2 R 4. But 4 does not divide 2, so 4 R 2. Thus, R is not symmetric.
% \item[(c)]  To see that R is transitive, let a, b, c be integers. Suppose that a R b and b R c. Thus,
% a divides b and b divides c so there exist integers k and l such that b = ak and c = bl. This
% gives c = bl = (ak)l = a(kl). Therefore, a divides c so a R c
% \end{itemize}

%--------------------------------------------------------%

\item 
Given the set $S =\{g,e,r,b,i,l\}$.
\begin{itemize}
\item[(a)] Describe how each subset of $S$ can be represented by using a 6 digit binary string
\item[(b)] Write down the string corresponding to the subset $\{g,r,l\}$ and the subset corresponding to the string 010101.
\item[(c)] What is the total number of subsets of S?
\end{itemize}
%----------------------------------------------------------------%

\end{enumerate}
\end{document}
